\documentclass[12pt]{article}
\usepackage{graphicx, verbatim}
\setlength{\textwidth}{6.5in}
\setlength{\textheight}{9in}
\setlength{\oddsidemargin}{0in}
\setlength{\evensidemargin}{0in}
\setlength{\topmargin}{-1.5cm}

\usepackage{Sweave}
\begin{document}
%\input{proposal-concordance}
\title{Prediction of wine quality from its physicochemical properties}
\author{\small{Sohom Mandal} \\
\small{Dept. of Civil and Environmental Engineering}\\
\small{Western University}}
\date{}
\maketitle


 \section{Introduction}
Wine is a luxary alcoholic beverage.A significant amount of business revenue is invloved with this industry.
The economical value depends on it's quality and quality of wine depends on physicochemical properties of
ingredients.
This analysis is important for vineyard to improve or predict  wine quality based on its physicochemical
characteristics.

For this study one red wine and one white wine datasets have been used were obtained from UCI Machine Learning
Repository (\cite{Cortez}).
The each dataset contains 12 variables, 11 among of them are explanatory variable and 1 variable is integer
catagorical variable.
There are 1599 samples for the red wine and 4898 samples for the white wine, no missing data was found.

\section{Preliminary Data Analysis}
The variables summary statistics for red wine are given in Tables \ref{tab:summary0}, \ref{tab:summary1} and
\ref{tab:summary2}. The summary statistics for white wine data are shown in Tables \ref{tab:summary3},
\ref{tab:summary4} and \ref{tab:summary5}. From summary statistics, we can see that the quality of red wine ranges from $3$ to $8$ and the quality of white wine data is between $3$ and $9.$
%~\ref{tab:coef1} and Table~\ref{tab:coef2} respectively.
\begin{center}
% latex table generated in R 3.0.2 by xtable 1.7-1 package
% Sat Apr 05 14:52:15 2014
\begin{table}[htbp]
\centering
\begin{tabular}{rllll}
  \hline
 & fixed.acidity & volatile.acidity &  citric.acid & residual.sugar \\
  \hline
1 & Min.   : 4.60   & Min.   :0.1200   & Min.   :0.000   & Min.   : 0.900   \\
  2 & 1st Qu.: 7.10   & 1st Qu.:0.3900   & 1st Qu.:0.090   & 1st Qu.: 1.900   \\
  3 & Median : 7.90   & Median :0.5200   & Median :0.260   & Median : 2.200   \\
  4 & Mean   : 8.32   & Mean   :0.5278   & Mean   :0.271   & Mean   : 2.539   \\
  5 & 3rd Qu.: 9.20   & 3rd Qu.:0.6400   & 3rd Qu.:0.420   & 3rd Qu.: 2.600   \\
  6 & Max.   :15.90   & Max.   :1.5800   & Max.   :1.000   & Max.   :15.500   \\
   \hline
\end{tabular}
\caption{Input variables 1 to 4 for the red wine data}
\label{tab:summary0}
\end{table}
\end{center}

\begin{center}
% latex table generated in R 3.0.2 by xtable 1.7-1 package
% Sat Apr 05 15:16:01 2014
\begin{table}[htbp]
\centering
\begin{tabular}{rllll}
  \hline
 &   chlorides & free.sulfur.dioxide & total.sulfur.dioxide &    density \\
  \hline
1 & Min.   :0.01200   & Min.   : 1.00   & Min.   :  6.00   & Min.   :0.9901   \\
  2 & 1st Qu.:0.07000   & 1st Qu.: 7.00   & 1st Qu.: 22.00   & 1st Qu.:0.9956   \\
  3 & Median :0.07900   & Median :14.00   & Median : 38.00   & Median :0.9968   \\
  4 & Mean   :0.08747   & Mean   :15.87   & Mean   : 46.47   & Mean   :0.9967   \\
  5 & 3rd Qu.:0.09000   & 3rd Qu.:21.00   & 3rd Qu.: 62.00   & 3rd Qu.:0.9978   \\
  6 & Max.   :0.61100   & Max.   :72.00   & Max.   :289.00   & Max.   :1.0037   \\
   \hline
\end{tabular}
\caption{Input variables 5 to 8 for the red wine data}
\label{tab:summary1}
\end{table}\end{center}

\begin{center}
% latex table generated in R 3.0.2 by xtable 1.7-1 package
% Sat Apr 05 15:21:30 2014
\begin{table}[htbp]
\centering
\begin{tabular}{rllll}
  \hline
 &       pH &   sulphates &    alcohol &    quality \\
  \hline
1 & Min.   :2.740   & Min.   :0.3300   & Min.   : 8.40   & Min.   :3.000   \\
  2 & 1st Qu.:3.210   & 1st Qu.:0.5500   & 1st Qu.: 9.50   & 1st Qu.:5.000   \\
  3 & Median :3.310   & Median :0.6200   & Median :10.20   & Median :6.000   \\
  4 & Mean   :3.311   & Mean   :0.6581   & Mean   :10.42   & Mean   :5.636   \\
  5 & 3rd Qu.:3.400   & 3rd Qu.:0.7300   & 3rd Qu.:11.10   & 3rd Qu.:6.000   \\
  6 & Max.   :4.010   & Max.   :2.0000   & Max.   :14.90   & Max.   :8.000   \\
   \hline
\end{tabular}
\caption{Input variables 9 to 12 for the red wine data}
\label{tab:summary2}
\end{table}\end{center}

\begin{center}
% latex table generated in R 3.0.2 by xtable 1.7-1 package
% Sat Apr 05 15:30:17 2014
\begin{table}[htbp]
\centering
\begin{tabular}{rllll}
  \hline
 & fixed.acidity & volatile.acidity &  citric.acid & residual.sugar \\
  \hline
1 & Min.   : 3.800   & Min.   :0.0800   & Min.   :0.0000   & Min.   : 0.600   \\
  2 & 1st Qu.: 6.300   & 1st Qu.:0.2100   & 1st Qu.:0.2700   & 1st Qu.: 1.700   \\
  3 & Median : 6.800   & Median :0.2600   & Median :0.3200   & Median : 5.200   \\
  4 & Mean   : 6.855   & Mean   :0.2782   & Mean   :0.3342   & Mean   : 6.391   \\
  5 & 3rd Qu.: 7.300   & 3rd Qu.:0.3200   & 3rd Qu.:0.3900   & 3rd Qu.: 9.900   \\
  6 & Max.   :14.200   & Max.   :1.1000   & Max.   :1.6600   & Max.   :65.800   \\
   \hline
\end{tabular}
\caption{Input variables 1 to 4 for the white wine data}
\label{tab:summary3}
\end{table}\end{center}

\begin{center}
% latex table generated in R 3.0.2 by xtable 1.7-1 package
% Sat Apr 05 15:30:17 2014
\begin{table}[htbp]
\centering
\begin{tabular}{rllll}
  \hline
 &   chlorides & free.sulfur.dioxide & total.sulfur.dioxide &    density \\
  \hline
1 & Min.   :0.00900   & Min.   :  2.00   & Min.   :  9.0   & Min.   :0.9871   \\
  2 & 1st Qu.:0.03600   & 1st Qu.: 23.00   & 1st Qu.:108.0   & 1st Qu.:0.9917   \\
  3 & Median :0.04300   & Median : 34.00   & Median :134.0   & Median :0.9937   \\
  4 & Mean   :0.04577   & Mean   : 35.31   & Mean   :138.4   & Mean   :0.9940   \\
  5 & 3rd Qu.:0.05000   & 3rd Qu.: 46.00   & 3rd Qu.:167.0   & 3rd Qu.:0.9961   \\
  6 & Max.   :0.34600   & Max.   :289.00   & Max.   :440.0   & Max.   :1.0390   \\
   \hline
\end{tabular}
\caption{Input variables 5 to 8 for the white wine data}
\label{tab:summary4}
\end{table}\end{center}

\begin{center}
% latex table generated in R 3.0.2 by xtable 1.7-1 package
% Sat Apr 05 15:30:17 2014
\begin{table}[htbp]
\centering
\begin{tabular}{rllll}
  \hline
 &       pH &   sulphates &    alcohol &    quality \\
  \hline
1 & Min.   :2.720   & Min.   :0.2200   & Min.   : 8.00   & Min.   :3.000   \\
  2 & 1st Qu.:3.090   & 1st Qu.:0.4100   & 1st Qu.: 9.50   & 1st Qu.:5.000   \\
  3 & Median :3.180   & Median :0.4700   & Median :10.40   & Median :6.000   \\
  4 & Mean   :3.188   & Mean   :0.4898   & Mean   :10.51   & Mean   :5.878   \\
  5 & 3rd Qu.:3.280   & 3rd Qu.:0.5500   & 3rd Qu.:11.40   & 3rd Qu.:6.000   \\
  6 & Max.   :3.820   & Max.   :1.0800   & Max.   :14.20   & Max.   :9.000   \\
   \hline
\end{tabular}
\caption{Input variables 9 to 12 for the white wine data}
\label{tab:summary5}
\end{table}\end{center}


\newpage



\section{Materials and Methods}
\subsection{LDA, KNN and Multinomial Logistic Regression}
In this section, we apply three different classification methods, LDA, KNN, Multinomial Logistic Regression
to both red and white wine data. LDA classifier is a common choice when the response variable has more than two categories. LDA method assumes each predictor $x_i,$ where $i=1 \,{\rm to}\, 11,$ has the Gaussian distribution and $X=(x_1,x_2,...,x_{11})$ follows the multivariate Gaussian distribution. It then uses the linear discriminant functions
to approximate the Bayes classifier.
Next, we try KNN method. Unlike LDA method, KNN method is a non-parametric method. KNN classifier first identifies $K$ nearest points to the certain point and then predict the conditional probability based on these points.
We also explore Multinomial Logistic Regression to our wine data. Multinomial Logistic Regression is similar to binary-case logistic regression. The difference is that instead of predicting probability of certain observation in each category in binary-case logistic regression, multinomial logistic regression takes one category as the base one, and then calculates the relative probability compared with the base category. In addition, we combine ridge regression with multinomial logistic regression in order to make better predictions. Ridge regression uses the $L_1$ penalty to shrink the parameter values and reduce dimensions. For this part, we employ glmnet R package to do analysis. Before we apply any method, the data are scaled first.


\subsection{K-means clustering, PCA and Regression Tree}
Acid is a measure of $H^{+}$ ions concentration in a solution, on the other hand pH of a solution measured by
$-\log[H^{+}]$ ion concentration. Actually if pH of a solution is less than 7, it's consider as acid or more than
7 is consider as base. PH and acidity are correlated among each other.  3 among 11 explanatory variables are acid
and other variables are also correlated with pH.
Therefore for this analysis pH has used as a dependent variable or predictand variable where other variables
(e.g. fixed acidity, volatile acidity, density etc) were independent variable or predictors. In this study, 2/3
of red wine data is used as a training data and rest is for testing purpose.

According to Kodur (2011) (\cite{Kour}), if pH is low $(e.g.<3.0)$, quality of wine will be better than if pH is high
$(e.g.>3.8)$ in the grape juice. On account of this condition, K-means clustering is used for clustering data.
The clusters are made on basis of pH.
As predictor data and response data are highly co-related with each other, to avoid multi-dimensionality and
multi-collinearity, Principal Component Analysis (PCA) has applied in this work. After PCA analysis, regression
tree is used for predicting wine quality. The categorical variable data is not used for this part of analysis.



\section{Results}

\subsection{Results of LDA, KNN and Multinomial Logistic Regression}
In this section, we present the results of LDA, KNN and Multinomial Logistic Regression on wine data.
After applying lda classier on red wine data, we obtain the confusion matrix as below Table \ref{tab:lda0}. The horizontal axis is the true classification and the vertical axis is the prediction. 
\begin{center}
% latex table generated in R 3.0.2 by xtable 1.7-1 package
% Fri Apr 04 23:39:36 2014
\begin{table}[htbp]
\centering
\begin{tabular}{rrrrrrr}
  \hline
 & 3 & 4 & 5 & 6 & 7 & 8 \\
  \hline
3 &   2 &   4 &   6 &   2 &   0 &   0 \\
  4 &   0 &   1 &   5 &   4 &   0 &   0 \\
  5 &   7 &  30 & 504 & 202 &  13 &   0 \\
  6 &   1 &  16 & 158 & 370 & 108 &   9 \\
  7 &   0 &   2 &   8 &  59 &  78 &   9 \\
  8 &   0 &   0 &   0 &   1 &   0 &   0 \\
   \hline
\end{tabular}
\caption{Red Wine: Confusion Matrix of LDA Method}
\label{tab:lda0}
\end{table}\end{center}
We use 10-fold cross-validation to calculate the prediction error on test data. As shown below, the test error rate of LDA method applying on the red wine data is:
\begin{Schunk}
\begin{Soutput}
[1] 0.4027517.
\end{Soutput}
\end{Schunk}
That's saying, by using LDA classifier, we can make around $1-0.4027517=0.5972483$ correct prediction on test data.
Similarly, we get results of white wine data. The confusion matrix of LDA classifier on white wine data shows as Table \ref{tab:lda1}.

\begin{center}
% latex table generated in R 3.0.2 by xtable 1.7-1 package
% Fri Apr 04 23:53:02 2014
\begin{table}[htbp]
\centering
\begin{tabular}{rrrrrrrr}
  \hline
 & 3 & 4 & 5 & 6 & 7 & 8 & 9 \\
  \hline
3 &   4 &   2 &   3 &   1 &   0 &   0 &   0 \\
  4 &   2 &  34 &  44 &  18 &   0 &   2 &   0 \\
  5 &   5 &  66 & 725 & 393 &  40 &   4 &   0 \\
  6 &   9 &  58 & 671 & 1598 & 595 & 103 &   1 \\
  7 &   0 &   3 &  13 & 185 & 245 &  66 &   4 \\
  8 &   0 &   0 &   1 &   0 &   0 &   0 &   0 \\
  9 &   0 &   0 &   0 &   3 &   0 &   0 &   0 \\
   \hline
\end{tabular}
\caption{White Wine: Confusion Matrix of LDA Method}
\label{tab:lda1}
\end{table}

\newpage
Again, we use 10-fold cross-validation to estimate the test error of LDA classifier on white wine data, which is shown below:
\begin{Schunk}
\begin{Soutput}
[1] 0.4687628.
\end{Soutput}
\end{Schunk}
\end{center}

For KNN method, first, we choose $k=1,$ where the first nearest point is identified to make prediction. We use 10-fold cross-validation method to estimate test errors. The test error for KNN classifier on red wine data is
\begin{Schunk}
\begin{Soutput}
[1] 0.1163227
\end{Soutput}
\end{Schunk}
and on white wine data is
 \begin{Schunk}
\begin{Soutput}
[1] 0.1075949.
\end{Soutput}
\end{Schunk}
 Varying the value of k, we get cross-validation errors as listed below. The first list shows the test errors for red wine and the second one indicates test errors for white wine. Here, we choose $k$ values from $1$ to $20.$ We visualize the test error with respect to different $k$ values, as indicated in Figures \ref{fig:knn0} and \ref{fig:knn1}. Based on Figures \ref{fig:knn0} and \ref{fig:knn1}, we can say that increasing the value of k will not improve the prediction too much.
Cross-validation errors of red wine data are:
\begin{Schunk}
\begin{Soutput}
 [1] 0.1163227 0.1663540 0.1500938 0.1638524 0.1644778 0.1694809 0.1682301
 [8] 0.1663540 0.1776110 0.1726079 0.1682301 0.1732333 0.1757348 0.1763602
[15] 0.1719825 0.1726079 0.1732333 0.1751094 0.1751094 0.1719825
\end{Soutput}
\end{Schunk}
Cross validation errors of white wine are:
\begin{Schunk}
\begin{Soutput}
 [1] 0.1075949 0.1455696 0.1380155 0.1351572 0.1380155 0.1443446 0.1463863
 [8] 0.1496529 0.1518987 0.1584320 0.1606778 0.1608820 0.1649653 0.1686403
[15] 0.1692528 0.1678236 0.1665986 0.1698653 0.1753777 0.1757860
\end{Soutput}
\end{Schunk}

\begin{figure} [htbp]
 \begin{center}
\includegraphics[height=2.0in,width=3in]{lda0.eps}
\caption{Red Wine: Varying the value of $p=k,$ the cv.error will change.}
 \label{fig:knn0}
\end{center}
\end{figure}

\begin{figure} [htbp]
 \begin{center}
\includegraphics[height=2.0in,width=3in]{lda1.eps}
\caption{White Wine: Varying the value of $p_0=k,$ the cv.error will change.}
 \label{fig:knn1}
\end{center}
\end{figure}

\newpage

For Multinomial Logistic Regression with $L_1$ penalty, the value of the tuning parameter $\lambda$ will affect the prediction outcomes. Here, we use cross-validation to estimate the best $\lambda$ value. As indicated in Figure \ref{fig:ridge1} and Figure \ref{fig:ridge2}, we can see that when the value of $\lambda$ changes, the cross-validation error will change. We choose the best $\lambda$ value by choosing the one such that the cross-validation error is minimized. For red wine data, the best $\lambda$ is
\begin{Schunk}
\begin{Soutput}
[1] 0.01891325
\end{Soutput}
\end{Schunk}
and for white wine data, the best $\lambda$ is
\begin{Schunk}
\begin{Soutput}
[1] 0.01675522.
\end{Soutput}
\end{Schunk}
The confusion matrix of red wine data and white wine data by using multinomial logistic regression are presented as Table \ref{tab:ridge1} and Table \ref{tab:ridge2} respectively. The correct prediction rate for red wine data and white wine data are 
\begin{Schunk}
\begin{Soutput}
[1] 0.63125
\end{Soutput}
\end{Schunk}
and 
\begin{Schunk}
\begin{Soutput}
[1] 0.5357289
\end{Soutput}
\end{Schunk}
respetively.

\begin{center}
% latex table generated in R 3.0.2 by xtable 1.7-1 package
% Sat Apr 05 11:46:08 2014
\begin{table}[htbp]
\centering
\begin{tabular}{rrrrrrr}
  \hline
 & 3 & 4 & 5 & 6 & 7 & 8 \\
  \hline
4 &   1 &   0 &   0 &   0 &   0 &   0 \\
  5 &   3 &  13 & 264 & 101 &   4 &   0 \\
  6 &   1 &   5 &  74 & 217 &  73 &   6 \\
  7 &   0 &   1 &   0 &   9 &  24 &   4 \\
   \hline
\end{tabular}
\caption{Red Wine: Confusion Matrix of Multinomial Logistic Regression with Penalty}
\label{tab:ridge1}
\end{table}\end{center}

\begin{center}
% latex table generated in R 3.0.2 by xtable 1.7-1 package
% Sat Apr 05 11:57:20 2014
\begin{table}[htbp]
\centering
\begin{tabular}{rrrrrrrr}
  \hline
 & 3 & 4 & 5 & 6 & 7 & 8 & 9 \\
  \hline
5 &   4 &  49 & 396 & 210 &  21 &   3 &   0 \\
  6 &   6 &  25 & 359 & 856 & 345 &  56 &   0 \\
  7 &   0 &   0 &   3 &  45 &  60 &   9 &   2 \\
   \hline
\end{tabular}
\caption{White Wine: Confusion Matrix of Multinomial Logistic Regression with Penalty}
\label{tab:ridge2}
\end{table}\end{center}

\begin{figure} [htbp]
 \begin{center}
\includegraphics[height=3.0in,width=3in]{ridge1.eps}
\caption{Red Wine: Varying the value of the tuning parameter $\lambda,$ the deviance will change}
 \label{fig:ridge1}
\end{center}
\end{figure}


\begin{figure} [htbp]
 \begin{center}
\includegraphics[height=3.0in,width=3in]{ridge2.eps}
\caption{White Wine: Varying the value of the tuning parameter $\lambda,$ the deviance will change}
 \label{fig:ridge2}
\end{center}
\end{figure}


\subsection{Results of K-means clustering, PCA and Regression Tree}
\textbf{K-means Clustering:} The thumb rule of finding number clusters is equal to $\sqrt{\frac{n}{2}}$  where n is
the number of data points. For red wine data set which is 28 and 50 for white wine data. But
optimum number of cluster is not known. The optimum number of clusters is decided on
basis of cluster validity index. This cluster algorithm gives classified cluster identification (ID) in
each runs which are used for computation of cluster validity index. In this case study, internal validity measures are used which measures
compactness, connectedness, and separation of clusters (\cite{Brock}). Three validation measures Dunn Index, Silhoutte Index
and connectivity are computed for each clusters for internal validity measures. Figure \ref{fig:cluster0} shows
plot of
various validity indices computed against the number of clusters used for clustering.

\begin{figure} [htbp]
 \begin{center}
\includegraphics[height=3.0in,width=4in]{clv.eps}
\caption{Cluster validity indexes}
 \label{fig:cluster0}
\end{center}
\end{figure}

For optimum clusters, Dunn Index and Silhouette Index should be maximized where connectivity should be minimized.
From the Figure \ref{fig:cluster0} and on account of Kour's theory (\cite{Kour}) optimal clusters for this case is taken as 3.
\begin{Schunk}
\begin{Sinput}
> k2=matrix(k1$center)
\end{Sinput}
\end{Schunk}

\begin{Schunk}
\begin{Soutput}
      [,1]
1 3.524732
2 3.130541
3 3.321922
\end{Soutput}
\end{Schunk}


After computing k=3 clusters, means of clusters are 3.52, 3.32, 3.13. It can be concluded, vectors in clusters
means 3.13 indicate excellent quality of wine than other two clusters because wine quality will be good if
pH is less.

\textbf{Principal Component Analysis (PCA):}
To reduce multicollinearity and dimensionality PCA is performed. PCA is used separately for training and testing
data. The benefit of PCA, it is possible to represent variability among dataset by using a small number of
principal components.  Number of principal component is decided based on percentage of total variability in
principal components. The percentage of total variance $(w_k)$ explained by kth principal component is given by:
$$w_k=\frac{100\,\lambda_k}{\sum_{m=1}^{M}\lambda_m},$$
where $\lambda_k$ is $kth$ eigenvalue and M is dimensionality of original dataset.
 In this study we consider 97\% as threshold (i.e., the total percentage of the variability carried by the
 selected principal components should be more than or equal to 97\% of the original data). Figure \ref{fig:pca}
 shows percentage of variance for training and testing data. It has found that first
 eight principal components represent more than 97\% variability of both training and testing datasets.

\begin{Schunk}
\begin{Sinput}
> per_var_tr
\end{Sinput}
\begin{Soutput}
 Comp.1  Comp.2  Comp.3  Comp.4  Comp.5  Comp.6  Comp.7  Comp.8  Comp.9
  25.95   44.46   59.32   72.48   81.97   88.16   93.68   97.03   98.77
Comp.10
 100.00
\end{Soutput}
\begin{Sinput}
> per_var_test
\end{Sinput}
\begin{Soutput}
 Comp.1  Comp.2  Comp.3  Comp.4  Comp.5  Comp.6  Comp.7  Comp.8  Comp.9
  26.47   46.92   63.28   73.77   82.37   88.53   93.15   96.79   98.75
Comp.10
 100.00
\end{Soutput}
\end{Schunk}


 \begin{figure} [htbp]
 \begin{center}
\includegraphics[height=3.0in,width=4in]{pca.eps}
\caption{Percentage of Total Variance Vs Principle Components}
 \label{fig:pca}
\end{center}
\end{figure}

Therefore we build regression trees based on first eight principal components of red wine data. It is found from
summary of regression tree fitting; five principal components (comp.1, comp.4, comp.5, comp.6, comp.7) are used
in constructing tree.

\begin{Schunk}
\begin{Sinput}
> summary(tree.fit)
\end{Sinput}
\begin{Soutput}
Regression tree:
tree(formula = pH ~ da[, 1] + da[, 2] + da[, 3] + da[, 4] + da[,
    5] + da[, 6] + da[, 7] + da[, 8], data = da)
Variables actually used in tree construction:
[1] "da[, 1]" "da[, 5]" "da[, 7]" "da[, 4]" "da[, 6]"
Number of terminal nodes:  10
Residual mean deviance:  0.01251 = 13.21 / 1056
Distribution of residuals:
     Min.   1st Qu.    Median      Mean   3rd Qu.      Max.
-0.388600 -0.075260 -0.004615  0.000000  0.069150  0.369100
\end{Soutput}
\end{Schunk}

\begin{figure} [htbp]
 \begin{center}
\includegraphics[height=3.0in,width=4in]{tree.eps}
\caption{Regression Tree of Test Data}
 \label{fig:tree}
\end{center}
\end{figure}
The tree indicates that lower value of component 1 $(da[,1]<1.039)$ (Figure \ref{fig:tree}) correspondent to worst wine quality.

\begin{Schunk}
\begin{Sinput}
> mean((tree.pred-da1)^2)
\end{Sinput}
\begin{Soutput}
[1] 12.09435
\end{Soutput}
\end{Schunk}
We use this regression tree to predict wine quality from test data.
MSE associated from tree is 12.099.
The square root of MSE is 3.47 which indicates the best wine quality we can get below or equal to this pH value.
Above this $(pH>3.47)$ the quality of wine will be deteriorated.


\begin{thebibliography}{s99}
\bibitem{Kour}Kour S., Effects of juice pH and potassium on juice and wine quality , and regulation of potassium in grapevines through rootstocks (Vitis ): a short review, J. Grapevine Res., 50(1) 1-6, 2011.

\bibitem{Brock}Brock G, Pihur V, Datta S, Datta S. clValid , an R package for cluster validation. J. Stat. Softw., 
1-32, 2011.

\bibitem{James}Gareth James, Daniela Witten, Trevor Hastie and Robert Tibshirani, {\it An Introduction to
Statistical Learning with Applications in R}, Springer, 2013.

\bibitem{Cortez}P. Cortez, A. Cerdeira, F. Almeida, T. Matos and J. Reis., Modeling wine preferences by data mining from physicochemical properties, In Decision Support Systems, Elsevier, 47(4):547-553, 2009.
\end{thebibliography}

\end{document}



